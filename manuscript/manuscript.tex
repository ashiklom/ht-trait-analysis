%%%%%%%%%%%%%%%%%%%%%%%%%%%%%%%%%%%%%%%%%%%%%%%%%%%%%%%%%%%%%%%%%%%%%%%%%%%%
% AGUJournalTemplate.tex: this template file is for articles formatted with LaTeX
%
% This file includes commands and instructions
% given in the order necessary to produce a final output that will
% satisfy AGU requirements, including customized APA reference formatting.
%
% You may copy this file and give it your
% article name, and enter your text.
%
%
% Step 1: Set the \documentclass
%
%

%% To submit your paper:
\documentclass[draft]{agujournal2019}
\usepackage{url} %this package should fix any errors with URLs in refs.
\usepackage{lineno}
\usepackage[inline]{trackchanges} %for better track changes. finalnew option will compile document with changes incorporated.
\usepackage{soul}
\linenumbers
%%%%%%%
% As of 2018 we recommend use of the TrackChanges package to mark revisions.
% The trackchanges package adds five new LaTeX commands:
%
%  \note[editor]{The note}
%  \annote[editor]{Text to annotate}{The note}
%  \add[editor]{Text to add}
%  \remove[editor]{Text to remove}
%  \change[editor]{Text to remove}{Text to add}
%
% complete documentation is here: http://trackchanges.sourceforge.net/
%%%%%%%

\draftfalse

%% Enter journal name below.
%% Choose from this list of Journals:
%
% JGR: Atmospheres
% JGR: Biogeosciences
% JGR: Earth Surface
% JGR: Oceans
% JGR: Planets
% JGR: Solid Earth
% JGR: Space Physics
% Global Biogeochemical Cycles
% Geophysical Research Letters
% Paleoceanography and Paleoclimatology
% Radio Science
% Reviews of Geophysics
% Tectonics
% Space Weather
% Water Resources Research
% Geochemistry, Geophysics, Geosystems
% Journal of Advances in Modeling Earth Systems (JAMES)
% Earth's Future
% Earth and Space Science
% Geohealth
%
% ie, \journalname{Water Resources Research}

\journalname{Enter journal name here}


\begin{document}

%% ------------------------------------------------------------------------ %%
%  Title
%
% (A title should be specific, informative, and brief. Use
% abbreviations only if they are defined in the abstract. Titles that
% start with general keywords then specific terms are optimized in
% searches)
%
%% ------------------------------------------------------------------------ %%

\title{Quantifying and partitioning uncertainties in retrievals of plant traits from imaging spectroscopy data}

%% ------------------------------------------------------------------------ %%
%
%  AUTHORS AND AFFILIATIONS
%
%% ------------------------------------------------------------------------ %%

% Authors are individuals who have significantly contributed to the
% research and preparation of the article. Group authors are allowed, if
% each author in the group is separately identified in an appendix.)

% List authors by first name or initial followed by last name and
% separated by commas. Use \affil{} to number affiliations, and
% \thanks{} for author notes.
% Additional author notes should be indicated with \thanks{} (for
% example, for current addresses).

% Example: \authors{A. B. Author\affil{1}\thanks{Current address, Antartica}, B. C. Author\affil{2,3}, and D. E.
% Author\affil{3,4}\thanks{Also funded by Monsanto.}}

\authors{Alexey N. Shiklomanov\affil{1}, Shawn Serbin\affil{2}, David R. Thompson\affil{3}, Philip G. Brodrick\affil{3}, and Benjamin Poulter\affil{1}}

\affiliation{1}{NASA Goddard Space Flight Center, Greenbelt, MD, USA}
\affiliation{2}{Brookhaven National Laboratory, Stony Brook, NY, USA}
\affiliation{3}{NASA Jet Propulsion Laboratory, Pasadena, CA, USA}

%% Corresponding Author:
% Corresponding author mailing address and e-mail address:

% (include name and email addresses of the corresponding author.  More
% than one corresponding author is allowed in this LaTeX file and for
% publication; but only one corresponding author is allowed in our
% editorial system.)

\correspondingauthor{Alexey N. Shiklomanov}{alexey.shiklomanov@nasa.gov}

%% Keypoints, final entry on title page.

%  List up to three key points (at least one is required)
%  Key Points summarize the main points and conclusions of the article
%  Each must be 140 characters or fewer with no special characters or punctuation and must be complete sentences

% Example:
% \begin{keypoints}
% \item	List up to three key points (at least one is required)
% \item	Key Points summarize the main points and conclusions of the article
% \item	Each must be 140 characters or fewer with no special characters or punctuation and must be complete sentences
% \end{keypoints}

\begin{keypoints}
\item enter point 1 here
\item enter point 2 here
\item enter point 3 here
\end{keypoints}

%% ------------------------------------------------------------------------ %%
%
%  ABSTRACT and PLAIN LANGUAGE SUMMARY
%
% A good Abstract will begin with a short description of the problem
% being addressed, briefly describe the new data or analyses, then
% briefly states the main conclusion(s) and how they are supported and
% uncertainties.

% The Plain Language Summary should be written for a broad audience,
% including journalists and the science-interested public, that will not have 
% a background in your field.
%
% A Plain Language Summary is required in GRL, JGR: Planets, JGR: Biogeosciences,
% JGR: Oceans, G-Cubed, Reviews of Geophysics, and JAMES.
% see http://sharingscience.agu.org/creating-plain-language-summary/)
%
%% ------------------------------------------------------------------------ %%

%% \begin{abstract} starts the second page

\begin{abstract}
[ enter your Abstract here ]
\end{abstract}

\section*{Plain Language Summary}
[ enter your Plain Language Summary here or delete this section]


%% ------------------------------------------------------------------------ %%
%
%  TEXT
%
%% ------------------------------------------------------------------------ %%

%%% Suggested section heads:
\section{Introduction}

Plant functional traits are measurable plant characteristics that are closely related to individual fitness and ecosystem function~\cite{violle2007let}.
Specifically, some traits (``effect traits'') directly affect ecosystem processes and are therefore valuable indicators of an ecosystem's current status, while others (``response traits'') are useful as predictors of how individuals and ecosystems will respond to future changes~\cite{lavorel2002predicting, funk2017revisiting}.
Traits provide more granular information about the impacts of environmental changes than species identity alone~\cite{cadotte2017should}.
% Plant traits are relevant to evaluating the functional diversity of an ecosystem~\cite{cadotte_2011_beyond_species, bouallala2020vegetation, bilton2010intraspecific, diaz2013functional, flynn2011functional, jager2015soil}.
Moreover, plant traits often serve as parameters in land surface and dynamic vegetation models~\cite{bonan_2012_reconciling, feng2013scale, harper2016improved, shiklomanov2020structure}.
Trait data are therefore valuable to multiple ecological subdisciplines.

Our best current understanding of plant functional traits comes from regional- to global scale trait databases such as TRY~\cite{kattge2020try} and BIEN~\cite{maitner2018bien}.
Although databases of in-situ trait measurements are larger and more open than ever before~\cite{kattge2020try}, large spatial and phylogenetic gaps and sampling biases in these datasets continue to pose a challenge in trait ecology~\cite{cornwell_2019_what}.
Fortunately, multiple decades of research have demonstrated that plant traits can be successfully estimated from remote measurements of canopy reflectance~\cite{cavender-bares2020remote, verrelst2019quantifying}.
Compared to field measurements, airborne and satellite measurements cover larger areas to reduce sampling biases, particularly in remote and rugged landscapes~\cite{jetz_2016_monitoring, marvin2014amazonian}.

The advantages of remote trait mapping come at a cost of much larger uncertainties.
Uncertainties in remote trait estimation originate from multiple sources.
First, the relationship between canopy reflectance and leaf traits is usually at least partially empirical and therefore uncertain.
Second, light leaving a canopy interacts significantly with the atmosphere before reaching the airborne or spaceborne instrument;
therefore, any trait retrieval workflow must first remove these atmospheric effects (i.e., ``atmospheric correction''), a process that is also highly uncertain~\cite{thompson2020quantifying}.
Third, the airborne or spaceborne instrument is never perfect, but rather is subject to measurement constraints such as a finite signal-to-noise ratio and imperfect calibration; this introduces additional uncertainty.

The NASA Surface Biology and Geology (SBG) designated observable—a planned global satellite imaging spectroscopy mission—therefore represents a significant new opportunity for studying the spatial patterns and temporal dynamics of plant traits worldwide.
An evaluation of the performance and trade-offs of SBG for trait estimation is therefore essential to the design of an SBG plant trait product.
In particular, because of the limited spatial domain of existing imaging spectroscopy data, trait retrieval algorithms have typically been developed and evaluated only for specific sites and under specific measurement conditions (e.g., instrument characteristics, atmospheric conditions, sun-sensor geometries), and the performance of these same algorithms at other sites and measurement conditions has not been investigated.

This manuscript is motivated by the following research questions:
(1) How large are uncertainties in trait retrievals from imaging spectroscopy?
(2) What are the largest sources of uncertainties in these retrievals?
To address these questions, we performed a simulation study that proceeded as follows:
First, starting from a known surface reflectance image, we used an atmospheric radiative transfer model to simulate satellite observations of radiance over a wide range of mission architectures and atmospheric conditions.
Second, for each simulated radiance measurement, we performed atmospheric correction to estimate surface reflectance (with uncertainty).
Finally, for each of these estimated reflectance spectra, we estimated plant trait values using existing models and evaluated the sources of variability and uncertainty across the simulations.

\section{Materials and Methods}\label{sec:methods}

\subsection{Input data}\label{subsec:input-data}

\emergencystretch 2em  % Fix problematic long texttt
We obtained atmospherically corrected surface reflectance imaging spectroscopy data from a National Ecological Observatory (NEON) Airborne Observing Platform (AOP) scene in the Upper East River basing of Colorado, USA (39.0$^{\circ}$ N, 106.9$^{\circ}$ W) as processed by \citeA{chadwick2020integrating} and available on Google Earth Engine under the collection \texttt{users/pgbrodrick/SFA/collections/ciacorn\_priority}.
A complete description of the data and processing methods are provided in \citeA{chadwick2020integrating}.
Briefly, the scene is a mosaic from several NEON AOP flights performed in summer 2018.
NEON AOP data have a spatial resolution of XXX m and an average spectral resolution of XXX nm over the range XXX nm to XXX nm. %TODO: Fill out these values.
The atmospheric correction routine was a custom application of the Atmospheric CORrection Now (ACORN‐6LX, Imspec LLC, Glendale) software package accounting for fine-scale variation in topography and sun-sensor geometry.

\fussy  % Return to defaults
We used Google Earth Engine to manually select and extract 500 vegetation pixels from this scene (Figure~\ref{fig:sitemap}).
The pixels exhibit substantial variability in brightness and spectral character:
green reflectance (550 nm) ranges from XXX to XXX;
near-infrared reflectance (1100 nm) ranges from XXX to XXX;
and hyperspectral normalized difference vegetation index (NDVI; XXX and XXX nm bands) varies from XXX to XXX (Figure~\ref{fig:input-summary}).

\begin{figure}[ht]
  \centering
  % \includegraphics[]{figures/...}
  \caption{\label{fig:sitemap} \
    True color composite of the input data location, with vegetation pixels highlighted.
  }
\end{figure}

\begin{figure}[ht]
  \centering
  % \includegraphics[options]{figures/...}
  \caption{\label{fig:input-summary} \
    Summary of 500 input reflectance spectra selected for this study.
    Main figure shows 10 example spectra (black lines) and the full range of the input reflectance (gray shading).
    Inset figure shows a histogram of hyperspectral normalized difference vegetation index (NDVI;\@red=XXXnm, NIR=XXXnm) across the 500 input spectra.
  }
\end{figure}


\subsection{Simulation workflow}\label{subsec:workflow}

To explore the contributions of different uncertainty sources, we performed a simulation workflow that consisted of three major steps:
First, starting from known surface reflectance (Section~\ref{subsec:input-data}), we simulated radiance as it would be measured by a satellite, accounting for different combinations of surface elevation, atmospheric composition, sun-sensor geometry, and detector characteristics (e.g., signal-to-noise ratio) (Section~\ref{subsec:}).
Second, for each of these simulated measurements, we performed atmospheric correction to estimate surface reflectance with uncertainty.
Third, for each of these estimated reflectance spectra, we estimated plant traits using existing models.

\subsubsection{Simulating measured radiance}\label{subsubsec:radiance}

Following \citeA{thompson2018optimal}, we model top-of-atmosphere radiance ($L_{TOA}$) as a function of incoming solar irradiance ($e_{0}$), surface reflectance ($\rho_{s}$), and three terms describing atmospheric optical properties:
the path reflectance ($\rho_{a}$), transmittance ($t$), and spherical albedo ($s$).
Assuming that the entire reflectance process is Lambertian, and accounting for the solar zenith angle ($\theta_{s}$), the expression for TOA radiance is:

\begin{linenomath*}
\begin{equation}
  L_{TOA} = \frac{e_{0}\cos(\theta_{s})}{\pi}\left( \rho_{a} + \frac{t \rho_{s}}{1 - s \rho_{s}} \right)
\end{equation}
\end{linenomath*}

The atmospheric optical properties ($\rho_{a}$, $t$, and $s$) all vary as a function of atmospheric composition and sun-sensor geometry.
To estimate these quantities, we used a hybrid approach (sRTMnet, Brodrick et al.~\emph{in review}) that combines
an atmospheric radiative transfer model (RTM) based on a computationally efficient successive orders of scattering approximation~\cite<Second Simulation of a Satellite Signal in the Solar Spectrum-Vector, 6SV, version 2.1,>{kotchenova2006validation}
with a neural network-based emulator of a more advanced RTM~\cite<MODTRAN,>{berk2017validation}.

In our experiment, we calculated these values over a factorial combination of atmospheric conditions, sun-sensor geometries, and additional mission characteristics (Table~\ref{tab:combinations}).
For atmospheric conditions and surface elevation, we considered ten combinations identified as representative of global conditions (REF).
These combinations included
aerosol optical depth at 550 nm (AOD) ranging from 0.09 to 0.802,
water vapor content ranging from 0.543 to 4.323 g cm$^{-2}$,
surface elevation ranging from 77 to 3571 m above sea level,
and two types of atmospheric profiles (REF): tropical and midlatitude summer (Table~\ref{tab:atmospheres}).
For sun-sensor geometries, we considered a factorial combination of observation dates and times (which, combined with observation location, determine the position of the sun), sensor zenith angle, and sensor relative (to the sun's position) azimuth angle.

\begin{table}
\caption{\label{tab:atmospheres} Globally-representative atmospheres used in this analysis.}
\centering
\begin{tabular}{p{0.15\textwidth}p{0.15\textwidth}p{0.15\textwidth}p{0.15\textwidth}p{0.3\textwidth}}
\hline
  Global relative abundance (\%)  & Surface elevation (m) & Aerosol optical depth & Water vapor (g cm$^{-2}$) & Profile type  \\
  \hline
  34.0 & 109 & 0.090 & 1.190 & Midlatitude summer \\
  13.8 & 105 & 0.123 & 2.716 & Midlatitude summer \\
  12.6 & 77 & 0.132 & 2.889 & Tropical \\
  9.1 & 1174 & 0.100 & 1.116 & Midlatitude summer \\
  8.7 & 263 & 0.350 & 1.748 & Midlatitude summer \\
  8.7 & 88 & 0.168 & 4.323 & Tropical \\
  5.5 & 233 & 0.445 & 2.936 & Tropical \\
  4.5 & 911 & 0.135 & 1.853 & Tropical \\
  2.4 & 227 & 0.802 & 2.611 & Tropical \\
  0.7 & 3571 & 0.138 & 0.543 & Midlatitude summer \\
\hline
\end{tabular}
\end{table}

\begin{table}
\caption{\label{tab:combinations} Factorial combination of inputs}
\centering
\begin{tabular}{p{0.2\linewidth}lp{0.4\linewidth}p{0.3\linewidth}}
  \hline
  Variable & Values & Comments \\
  \hline
  Atmosphere & \emph{See Table~\ref{tab:atmospheres}} & \\
  Date & March 20, June 21, September 22, December 21 & Northern hemisphere equinoxes and solstices
  Local time (24H) & 9:30, 10:00, 10:30, 12, 15:00 & Spaced over the day, with denser sampling around typical ``A-train'' satellite overpass around 10:30AM \\
  Sensor zenith (degrees) & 0, 10, 30  & \\
  Sensor relative azimuth (degrees) & 0, 30, 60, 90, 180 & \\
  \hline
  Total grid size & \multicolumn{2}{l}{64,800} \\
  \hline
\end{tabular}
\end{table}

\subsubsection{Atmospheric correction via optimal estimation}\label{subsubsec:isofit}

We used optimal estimation as implemented in the ``Isofit'' Python library~\cite{thompson2019optimal,thompson2018optimal}.
As the underlying atmospheric RTM, we used the same hybrid sRTMnet approach as described in Section~\ref{subsubsec:radiance}.

\subsubsection{Estimating plant functional traits}\label{subsubsec:traits}

Algorithms from \citeA{singh2015imaging}.
Algorithms from \citeA{chadwick2020integrating}.

\subsection{Analysis}\label{subsec:analysis}

We quantified the contributions of each uncertainty source to both estimates of surface reflectance and the final trait estimates.
Partitioning uncertainty.

Quantifying retrieval accuracy.
Quantifying bias.

\section{Results}\label{sec:results}

\section{Discussion}\label{sec:discussion}

\section{Conclusions}

%%

%  Numbered lines in equations:
%  To add line numbers to lines in equations,
%  \begin{linenomath*}
%  \begin{equation}
%  \end{equation}
%  \end{linenomath*}



%% Enter Figures and Tables near as possible to where they are first mentioned:
%
% DO NOT USE \psfrag or \subfigure commands.
%
% Figure captions go below the figure.
% Table titles go above tables;  other caption information
%  should be placed in last line of the table, using
% \multicolumn2l{$^a$ This is a table note.}
%
%----------------
% EXAMPLE FIGURES
%
% \begin{figure}
% \includegraphics{example.png}
% \caption{caption}
% \end{figure}
%
% Giving latex a width will help it to scale the figure properly. A simple trick is to use \textwidth. Try this if large figures run off the side of the page.
% \begin{figure}
% \noindent\includegraphics[width=\textwidth]{anothersample.png}
%\caption{caption}
%\label{pngfiguresample}
%\end{figure}
%
%
% If you get an error about an unknown bounding box, try specifying the width and height of the figure with the natwidth and natheight options. This is common when trying to add a PDF figure without pdflatex.
% \begin{figure}
% \noindent\includegraphics[natwidth=800px,natheight=600px]{samplefigure.pdf}
%\caption{caption}
%\label{pdffiguresample}
%\end{figure}
%
%
% PDFLatex does not seem to be able to process EPS figures. You may want to try the epstopdf package.
%

%
% ---------------
% EXAMPLE TABLE
%
% \begin{table}
% \caption{Time of the Transition Between Phase 1 and Phase 2$^{a}$}
% \centering
% \begin{tabular}{l c}
% \hline
%  Run  & Time (min)  \\
% \hline
%   $l1$  & 260   \\
%   $l2$  & 300   \\
%   $l3$  & 340   \\
%   $h1$  & 270   \\
%   $h2$  & 250   \\
%   $h3$  & 380   \\
%   $r1$  & 370   \\
%   $r2$  & 390   \\
% \hline
% \multicolumn{2}{l}{$^{a}$Footnote text here.}
% \end{tabular}
% \end{table}

%% SIDEWAYS FIGURE and TABLE
% AGU prefers the use of {sidewaystable} over {landscapetable} as it causes fewer problems.
%
% \begin{sidewaysfigure}
% \includegraphics[width=20pc]{figsamp}
% \caption{caption here}
% \label{newfig}
% \end{sidewaysfigure}
%
%  \begin{sidewaystable}
%  \caption{Caption here}
% \label{tab:signif_gap_clos}
%  \begin{tabular}{ccc}
% one&two&three\\
% four&five&six
%  \end{tabular}
%  \end{sidewaystable}

%% If using numbered lines, please surround equations with \begin{linenomath*}...\end{linenomath*}
%\begin{linenomath*}
%\begin{equation}
%y|{f} \sim g(m, \sigma),
%\end{equation}
%\end{linenomath*}

%%% End of body of article

%%%%%%%%%%%%%%%%%%%%%%%%%%%%%%%%
%% Optional Appendix goes here
%
% The \appendix command resets counters and redefines section heads
%
% After typing \appendix
%
%\section{Here Is Appendix Title}
% will show
% A: Here Is Appendix Title
%
%\appendix
%\section{Here is a sample appendix}

%%%%%%%%%%%%%%%%%%%%%%%%%%%%%%%%%%%%%%%%%%%%%%%%%%%%%%%%%%%%%%%%
%
% Optional Glossary, Notation or Acronym section goes here:
%
%%%%%%%%%%%%%%
% Glossary is only allowed in Reviews of Geophysics
%  \begin{glossary}
%  \term{Term}
%   Term Definition here
%  \term{Term}
%   Term Definition here
%  \term{Term}
%   Term Definition here
%  \end{glossary}

%
%%%%%%%%%%%%%%
% Acronyms
%   \begin{acronyms}
%   \acro{Acronym}
%   Definition here
%   \acro{EMOS}
%   Ensemble model output statistics
%   \acro{ECMWF}
%   Centre for Medium-Range Weather Forecasts
%   \end{acronyms}

%
%%%%%%%%%%%%%%
% Notation
%   \begin{notation}
%   \notation{$a+b$} Notation Definition here
%   \notation{$e=mc^2$}
%   Equation in German-born physicist Albert Einstein's theory of special
%  relativity that showed that the increased relativistic mass ($m$) of a
%  body comes from the energy of motion of the body—that is, its kinetic
%  energy ($E$)—divided by the speed of light squared ($c^2$).
%   \end{notation}




%%%%%%%%%%%%%%%%%%%%%%%%%%%%%%%%%%%%%%%%%%%%%%%%%%%%%%%%%%%%%%%%
%
%  ACKNOWLEDGMENTS
%
% The acknowledgments must list:
%
% >>>>	A statement that indicates to the reader where the data
% 	supporting the conclusions can be obtained (for example, in the
% 	references, tables, supporting information, and other databases).
%
% 	All funding sources related to this work from all authors
%
% 	Any real or perceived financial conflicts of interests for any
%	author
%
% 	Other affiliations for any author that may be perceived as
% 	having a conflict of interest with respect to the results of this
% 	paper.
%
%
% It is also the appropriate place to thank colleagues and other contributors.
% AGU does not normally allow dedications.


\acknowledgments
\emergencystretch 2em
The code and data for performing this analysis is permanently archived at XXXXX. % TODO
Both Isofit and the Hypertrace wrapper script are publicly developed on GitHub at https://github.com/isofit/isofit.
This work was supported by the NASA Surface Biology and Geology (SBG) mission study.
The authors declare no conflicts of interest.

\fussy

%% ------------------------------------------------------------------------ %%
%% References and Citations

%%%%%%%%%%%%%%%%%%%%%%%%%%%%%%%%%%%%%%%%%%%%%%%
%
% \bibliography{<name of your .bib file>} don't specify the file extension
%
% don't specify bibliographystyle
%%%%%%%%%%%%%%%%%%%%%%%%%%%%%%%%%%%%%%%%%%%%%%%

\bibliography{zotero-library}



%Reference citation instructions and examples:
%
% Please use ONLY \cite and \citeA for reference citations.
% \cite for parenthetical references
% ...as shown in recent studies (Simpson et al., 2019)
% \citeA for in-text citations
% ...Simpson et al. (2019) have shown...
%
%
%...as shown by \citeA{jskilby}.
%...as shown by \citeA{lewin76}, \citeA{carson86}, \citeA{bartoldy02}, and \citeA{rinaldi03}.
%...has been shown \cite{jskilbye}.
%...has been shown \cite{lewin76,carson86,bartoldy02,rinaldi03}.
%... \cite <i.e.>[]{lewin76,carson86,bartoldy02,rinaldi03}.
%...has been shown by \cite <e.g.,>[and others]{lewin76}.
%
% apacite uses < > for prenotes and [ ] for postnotes
% DO NOT use other cite commands (e.g., \citet, \citep, \citeyear, \nocite, \citealp, etc.).
%



\end{document}



More Information and Advice:

%% ------------------------------------------------------------------------ %%
%
%  SECTION HEADS
%
%% ------------------------------------------------------------------------ %%

% Capitalize the first letter of each word (except for
% prepositions, conjunctions, and articles that are
% three or fewer letters).

% AGU follows standard outline style; therefore, there cannot be a section 1 without
% a section 2, or a section 2.3.1 without a section 2.3.2.
% Please make sure your section numbers are balanced.
% ---------------
% Level 1 head
%
% Use the \section{} command to identify level 1 heads;
% type the appropriate head wording between the curly
% brackets, as shown below.
%
%An example:
%\section{Level 1 Head: Introduction}
%
% ---------------
% Level 2 head
%
% Use the \subsection{} command to identify level 2 heads.
%An example:
%\subsection{Level 2 Head}
%
% ---------------
% Level 3 head
%
% Use the \subsubsection{} command to identify level 3 heads
%An example:
%\subsubsection{Level 3 Head}
%
%---------------
% Level 4 head
%
% Use the \subsubsubsection{} command to identify level 3 heads
% An example:
%\subsubsubsection{Level 4 Head} An example.
%
%% ------------------------------------------------------------------------ %%
%
%  IN-TEXT LISTS
%
%% ------------------------------------------------------------------------ %%
%
% Do not use bulleted lists; enumerated lists are okay.
% \begin{enumerate}
% \item
% \item
% \item
% \end{enumerate}
%
%% ------------------------------------------------------------------------ %%
%
%  EQUATIONS
%
%% ------------------------------------------------------------------------ %%

% Single-line equations are centered.
% Equation arrays will appear left-aligned.

Math coded inside display math mode \[ ...\]
 will not be numbered, e.g.,:
 \[ x^2=y^2 + z^2\]

 Math coded inside \begin{equation} and \end{equation} will
 be automatically numbered, e.g.,:
 \begin{equation}
 x^2=y^2 + z^2
 \end{equation}


% To create multiline equations, use the
% \begin{eqnarray} and \end{eqnarray} environment
% as demonstrated below.
\begin{eqnarray}
  x_{1} & = & (x - x_{0}) \cos \Theta \nonumber \\
        && + (y - y_{0}) \sin \Theta  \nonumber \\
  y_{1} & = & -(x - x_{0}) \sin \Theta \nonumber \\
        && + (y - y_{0}) \cos \Theta.
\end{eqnarray}

%If you don't want an equation number, use the star form:
%\begin{eqnarray*}...\end{eqnarray*}

% Break each line at a sign of operation
% (+, -, etc.) if possible, with the sign of operation
% on the new line.

% Indent second and subsequent lines to align with
% the first character following the equal sign on the
% first line.

% Use an \hspace{} command to insert horizontal space
% into your equation if necessary. Place an appropriate
% unit of measure between the curly braces, e.g.
% \hspace{1in}; you may have to experiment to achieve
% the correct amount of space.


%% ------------------------------------------------------------------------ %%
%
%  EQUATION NUMBERING: COUNTER
%
%% ------------------------------------------------------------------------ %%

% You may change equation numbering by resetting
% the equation counter or by explicitly numbering
% an equation.

% To explicitly number an equation, type \eqnum{}
% (with the desired number between the brackets)
% after the \begin{equation} or \begin{eqnarray}
% command.  The \eqnum{} command will affect only
% the equation it appears with; LaTeX will number
% any equations appearing later in the manuscript
% according to the equation counter.
%

% If you have a multiline equation that needs only
% one equation number, use a \nonumber command in
% front of the double backslashes (\\) as shown in
% the multiline equation above.

% If you are using line numbers, remember to surround
% equations with \begin{linenomath*}...\end{linenomath*}

%  To add line numbers to lines in equations:
%  \begin{linenomath*}
%  \begin{equation}
%  \end{equation}
%  \end{linenomath*}
